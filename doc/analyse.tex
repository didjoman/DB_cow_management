%%%%%%%%%%%%%%%%%%%%%%%%%%%%%%%%%%%%%%%%%%%%%%%%%%%%%%%%%%%%%%%%%%%%%
% LaTeX Template: Project Titlepage Modified (v 0.1) by rcx
%
% Original Source: http://www.howtotex.com
% Date: February 2014
% 
% This is a title page template which be used for articles & reports.
% 
% This is the modified version of the original Latex template from
% aforementioned website.
% 
%%%%%%%%%%%%%%%%%%%%%%%%%%%%%%%%%%%%%%%%%%%%%%%%%%%%%%%%%%%%%%%%%%%%%%

\documentclass[12pt]{report}
\usepackage[a4paper]{geometry}
\usepackage[utf8]{inputenc}
\usepackage[myheadings]{fullpage}
\usepackage{fancyhdr}
\usepackage{lastpage}
\usepackage{graphicx, wrapfig, subcaption, setspace, booktabs}
\usepackage[T1]{fontenc}
\usepackage[font=small, labelfont=bf]{caption}
\usepackage{fourier}
\usepackage[protrusion=true, expansion=true]{microtype}
\usepackage[french]{babel}
\usepackage{sectsty}
\usepackage{url, lipsum}
\usepackage{amsmath}
\usepackage{lscape}

% [begin] Numérotation des sections /subsec / subsubsec 
\usepackage{titlesec}
\titleformat{\chapter}
  {\Large\bfseries} % format
  {}                % label
  {0pt}             % sep
  {\huge}           % before-code

\renewcommand \thesection{\Roman{section}}
\renewcommand \thesubsection{\Roman{section}.\arabic{subsection}}
\renewcommand \thesubsubsection{\Roman{section}.\arabic{subsection}.\arabic{subsubsection}}
% [/end] Numérotation des sections /subsec / subsubsec 

% [begin] arrows with 2 heads
\usepackage{amssymb}
\usepackage{graphicx}

\newcommand\twoheaduparrow{\mathrel{\rotatebox[origin=c]{90}{$\twoheadrightarrow$}}}
% [/end]arrows with 2 heads

\newcommand{\HRule}[1]{\rule{\linewidth}{#1}}
\onehalfspacing
\setcounter{tocdepth}{5}
\setcounter{secnumdepth}{5}

%-------------------------------------------------------------------------------
% HEADER & FOOTER
%-------------------------------------------------------------------------------
\pagestyle{fancy}
\fancyhf{}
\setlength\headheight{15pt}
\fancyhead[L]{A.Rupp, R.Laguerre, F.Perroud, A.Bel Alonso}
\fancyhead[R]{2A, Projet Bases de Données}
\fancyfoot[R]{Page \thepage\ of \pageref{LastPage}}
%-------------------------------------------------------------------------------
% TITLE PAGE
%-------------------------------------------------------------------------------

\begin{document}

\title{ \normalsize \textsc{Projet Bases de Données}
		\\ [2.0cm]
		\HRule{0.5pt} \\
		\LARGE \textbf{\uppercase{Gestion d'une coopérative laitière}}
		\HRule{2pt} \\ [0.5cm]
		\normalsize \today \vspace*{5\baselineskip}}

\date{}

\author{
		Raphael Laguerre,\\ Alexandre Rupp,\\ Florian Perroud,\\ Andres Bel Alonso }

\maketitle
\tableofcontents
\newpage

%-------------------------------------------------------------------------------
% Section title formatting
\sectionfont{\scshape}
%-------------------------------------------------------------------------------

%-------------------------------------------------------------------------------
% BODY
%-------------------------------------------------------------------------------

\section*{Introduction}
%\lipsum

\section{Analyse des dépendances et contraintes}
\subsection{Dépendances fonctionnelles}
\noindent
NumSiren $\rightarrow$ NomExploitation, typeExploitation, nomCommune \\
NumSecuSociale $\rightarrow$ nomEleveur, prenomEleveur, adresseEleveur \\
NumVache $\rightarrow$ NomVache, race, rangLactation \\
NumVache, Date $\rightarrow$ NumEvenement \\
NumEvenement $\rightarrow$ NumVache, Date \textit{(ou on met une contrainte)} \\
NumEvenementProd $\rightarrow$ qteLait \\
NumEvenementMaladie $\rightarrow$ nomMaladie \\
NumEvenementVente $\rightarrow$ NumEleveur, NumVache, prixVache, vente

\subsection{Contraintes de valeur}
\noindent
typeExploitation $\in \lbrace$ GAEC, EARL $\rbrace$ \\
race $\in \lbrace$ Montbeliarde, Prim'Holstein $\rbrace$ \\
rangLactation $\in \llbracket$0, 5$\rrbracket$ \\
Ext(NumEvenementProd) $\subseteq$ Ext(NumEvenement) \\
Ext(NumEvenementMaladie)$\subseteq$ Ext(NumEvenement) \\
Ext(NumEvenementVente) $\subseteq$ Ext(NumEvenement) \\
Ext(NumEvenementVelage) $\subseteq$ Ext(NumEvenement) \\
Ext(NumEvenementDeces) $\subseteq$ Ext(NumEvenement) 

\subsection{Contraintes de multiplicité}
\noindent
NumSiren {\Large$\twoheadrightarrow$} NumEleveur (1 ou +) \\
NumSiren {\Large$\twoheadrightarrow$} NumVache \\
NumVelage {\Large$\twoheadrightarrow$} NumVache (1 ou +) \\
NumEvenementVente {\Large$\twoheadrightarrow$} NumEleveur (0 ou 1) \\

\subsection{Autres contraintes}
Une vache malade ne peut pas produire de lait. \\

\begin{landscape}

\section{Modèle Entité/Association}

\begin{figure}[!h]
\centering
\includegraphics[height=.75\textwidth]{./uml/schema.png}
\caption{Schéma Entité/Association UML.}
\end{figure}

\end{landscape}

\section{Modèle relationnel}
// TODO

\end{document}

%-------------------------------------------------------------------------------
% SNIPPETS
%-------------------------------------------------------------------------------

%\begin{figure}[!ht]
%	\centering
%	\includegraphics[width=0.8\textwidth]{file_name}
%	\caption{}
%	\centering
%	\label{label:file_name}
%\end{figure}

%\begin{figure}[!ht]
%	\centering
%	\includegraphics[width=0.8\textwidth]{graph}
%	\caption{Blood pressure ranges and associated level of hypertension (American Heart Association, 2013).}
%	\centering
%	\label{label:graph}
%\end{figure}

%\begin{wrapfigure}{r}{0.30\textwidth}
%	\vspace{-40pt}
%	\begin{center}
%		\includegraphics[width=0.29\textwidth]{file_name}
%	\end{center}
%	\vspace{-20pt}
%	\caption{}
%	\label{label:file_name}
%\end{wrapfigure}

%\begin{wrapfigure}{r}{0.45\textwidth}
%	\begin{center}
%		\includegraphics[width=0.29\textwidth]{manometer}
%	\end{center}
%	\caption{Aneroid sphygmomanometer with stethoscope (Medicalexpo, 2012).}
%	\label{label:manometer}
%\end{wrapfigure}

%\begin{table}[!ht]\footnotesize
%	\centering
%	\begin{tabular}{cccccc}
%	\toprule
%	\multicolumn{2}{c} {Pearson's correlation test} & \multicolumn{4}{c} {Independent t-test} \\
%	\midrule	
%	\multicolumn{2}{c} {Gender} & \multicolumn{2}{c} {Activity level} & \multicolumn{2}{c} {Gender} \\
%	\midrule
%	Males & Females & 1st level & 6th level & Males & Females \\
%	\midrule
%	\multicolumn{2}{c} {BMI vs. SP} & \multicolumn{2}{c} {Systolic pressure} & \multicolumn{2}{c} {Systolic Pressure} \\
%	\multicolumn{2}{c} {BMI vs. DP} & \multicolumn{2}{c} {Diastolic pressure} & \multicolumn{2}{c} {Diastolic pressure} \\
%	\multicolumn{2}{c} {BMI vs. MAP} & \multicolumn{2}{c} {MAP} & \multicolumn{2}{c} {MAP} \\
%	\multicolumn{2}{c} {W:H ratio vs. SP} & \multicolumn{2}{c} {BMI} & \multicolumn{2}{c} {BMI} \\
%	\multicolumn{2}{c} {W:H ratio vs. DP} & \multicolumn{2}{c} {W:H ratio} & \multicolumn{2}{c} {W:H ratio} \\
%	\multicolumn{2}{c} {W:H ratio vs. MAP} & \multicolumn{2}{c} {\% Body fat} & \multicolumn{2}{c} {\% Body fat} \\
%	\multicolumn{2}{c} {} & \multicolumn{2}{c} {Height} & \multicolumn{2}{c} {Height} \\
%	\multicolumn{2}{c} {} & \multicolumn{2}{c} {Weight} & \multicolumn{2}{c} {Weight} \\
%	\multicolumn{2}{c} {} & \multicolumn{2}{c} {Heart rate} & \multicolumn{2}{c} {Heart rate} \\
%	\bottomrule
%	\end{tabular}
%	\caption{Parameters that were analysed and related statistical test performed for current study. BMI - body mass index; SP - systolic pressure; DP - diastolic pressure; MAP - mean arterial pressure; W:H ratio - waist to hip ratio.}
%	\label{label:tests}
%\end{table}